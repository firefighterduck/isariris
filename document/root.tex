\documentclass[11pt,a4paper]{article}
\usepackage[T1]{fontenc}
\usepackage{isabelle,isabellesym}

% further packages required for unusual symbols (see also
% isabellesym.sty), use only when needed

\usepackage{amssymb}
  %for \<leadsto>, \<box>, \<diamond>, \<sqsupset>, \<mho>, \<Join>,
  %\<lhd>, \<lesssim>, \<greatersim>, \<lessapprox>, \<greaterapprox>,
  %\<triangleq>, \<yen>, \<lozenge>

%\usepackage{eurosym}
  %for \<euro>

%\usepackage[only,bigsqcap,bigparallel,fatsemi,interleave,sslash]{stmaryrd}
  %for \<Sqinter>, \<Parallel>, \<Zsemi>, \<Parallel>, \<sslash>

%\usepackage{eufrak}
  %for \<AA> ... \<ZZ>, \<aa> ... \<zz> (also included in amssymb)

%\usepackage{textcomp}
  %for \<onequarter>, \<onehalf>, \<threequarters>, \<degree>, \<cent>,
  %\<currency>

\usepackage{amsmath}
\usepackage{float}
\usepackage{url}
\usepackage{graphics}
% this should be the last package used
\usepackage{pdfsetup}

% urls in roman style, theory text in math-similar italics
\urlstyle{rm}
\isabellestyle{it}

% for uniform font size
%\renewcommand{\isastyle}{\isastyleminor}

\begin{document}

\title{IsarIris}
\author{Florian Sextl}
\date{Technical University of Munich\\[\baselineskip] \today}
\maketitle

\tableofcontents

We present a formalization of the influential paper 'A Decidable Fragment of
  Separation Logic' by Berdine et al \cite{JoshBerdine.2004}.

\includegraphics{session_graph}

\newpage

\section{Automation Concepts}
\begin{itemize}
\item Basic conceptual difference to Coq: no named hypotheses but moving of subterms
\item More or less explicit coercion of upred_holds and entailments, for goals by dedicated method,
  for theorems by custom attribute
\item automation might sometimes require reordering subgoals/searching for subgoals
\item separate handling of lhs and rhs rule application
\item no proof context like in IPM but implicit duplication where possible
\item different moving approaches
\item by commutativity: might require checking first, works on both sides, to work for arbitrary 
  large terms either moving bigger subterms or on-the-fly generation of comm lemmata, could be 
  optimized with moving lemmata "from x to head", can't handle wrappers
\item splitting based: can move several terms on lhs at once, can not move out of some wrappers, 
  requires I-pattern, can be done with "logic programming", requires lots of emps, not ordered,
  not guaranteed to move all parts or at most one per subpattern
\item framing: works only for rhs, can move out of most wrappers, based on "logic programming",
  very simple but powerfull
\item rule application for hypotheses: split to get lhs of rule, apply indirectly, can also be used to
  apply wands from other hypotheses
\item handling of existantial quantifiers easy, but instantiation only one term per call
\item pull all pure hypotheses into context, try to solve pure goals from context/simp
\item eliminate modalities after rule application with nested "logic programming"
\item translate type class approach to predicates on iprops, instances to lemmata, instance search
  to "logic programming" by rule application of instance lemmata
\item problems: can't rewrite terms/unify with unfolding/symbolic execution, matching with subgoal
  fixation breaks a few nice patterns, duplication handling can be involved
\item most automization can be split into three steps: searching for/moving requirements, applying 
  some resoning principle/rule, some possible cleanup
\item interactive guidance with patterns often necessary (otherwise either specialized ML methods or
  just failure if information from user is required, i.e. smart reasoning), patterns can be quite 
  generic but must be unique, otherwise position dependant
\item steps in the background: moving, reordering subgoals, removing emp, applying associativity,
  coercion (upred_holds/entailment, upred_valid/upred_pure valid)
\item subgoals often with schematic variables that might connect several subgoals, important to make
  reasoning less user dependant, often implicitely unified by rules (e.g. splitting)
\item some rules migth come with more premises, they are then new subgoal and can either be handled 
  in a unfied way if the rule is known or must be solved by the user
\item simplifications on the lhs are mostly allowed/won't break anything (due to the Iris base logic
  being intuitionistic), on the rhs must be explicitely done by the user as this could break further
  reasoning steps
\end{itemize}

% use vertical space instead of indenting paragraphs
\parindent 0pt
\parskip 0.9ex

% include generated text of all theories
\input{session}

\bibliographystyle{acm}
\bibliography{root}

\end{document}
